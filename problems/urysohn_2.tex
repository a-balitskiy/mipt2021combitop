\documentclass[12pt]{article}

\pagestyle{empty}
\usepackage[utf8]{inputenc}
\usepackage[T2A]{fontenc}
\usepackage[russian]{babel}
\usepackage{cmap}
\usepackage{amsthm}
\usepackage{amsmath}
\usepackage{amssymb}
%\usepackage{units}
%\usepackage{fancyhdr}
%\usepackage{forloop}
\usepackage[pdftex]{graphicx}
\relpenalty=10000
\binoppenalty=10000

\renewcommand{\baselinestretch}{1.0}
\renewcommand\normalsize{\sloppypar}

\setlength{\topmargin}{-0.5in}
\setlength{\textheight}{9.1in}
\setlength{\oddsidemargin}{-0.4in}
\setlength{\evensidemargin}{-0.4in}
\setlength{\textwidth}{7in}
\setlength{\parindent}{0ex}
\setlength{\parskip}{1ex}

\newtheorem{theorem}{Теорема}
\newtheorem*{theorem*}{Теорема}
\newtheorem{lemma}{Лемма}
\newtheorem*{lemma*}{Лемма}
\newtheorem{problem}{Задача}

\newcommand{\HRule}{\rule{\linewidth}{0.5mm}}
\newcommand{\divrus}{\mathop{\raisebox{-2pt}{\vdots}}}

\renewcommand{\le}{\leqslant}
\renewcommand{\ge}{\geqslant}

\DeclareMathOperator{\gcdrus}{\text{НОД}}
\DeclareMathOperator{\lcmrus}{\text{НОК}}
\DeclareMathOperator{\maj}{maj}
\DeclareMathOperator{\per}{per}
\DeclareMathOperator{\UW}{UW}
\DeclareMathOperator{\dist}{dist}
\DeclareMathOperator{\diam}{diam}

\def\ra{\rightarrow}
\def\Ra{\Rightarrow}
\def\ov{\overline}
\def\CC{\mathbb{C}}
\def\RR{\mathbb{R}}
\def\NN{\mathbb{N}}
\def\QQ{\mathbb{Q}}
\def\ZZ{\mathbb{Z}}
\def\eps{\varepsilon}

\begin{document}

{
\HRule
 \normalfont \Large \center
   Листочек 2. Поперечник Урысона

\HRule
}
Напомню, что $d$-мерный поперечник Урысона компактного метрического пространства $X$ может быть определён одним из следующих эквивалентных способов.

\[
\UW_d(X) = \inf\limits_{\substack{\bigcup U_i = X \\ \text{mult.} \{U_i\} \le d+1}} \sup\limits_{i} \diam(U_i),
\]
где инфимум берётся по открытым покрытиям кратности не более $d+1$.

\[
\UW_d(X) = \inf\limits_{\substack{p: X \to Z \\ \dim Z \le d}} \sup\limits_{z \in Z} \diam(p^{-1}(z)),
\]
где инфимум берётся по всем непрерывным отображениям $p$ в метризуемые пространства размерности не более $d$.

Обозначения в задачах:
\begin{itemize}
  \item $B^n$ обозначает шар единичного радиуса в $\mathbb{R}^n$;
  \item $\square^n$ обозначает куб с единичным ребром в $\mathbb{R}^n$;
  \item $\triangle^n$ обозначает правильный $n$-мерный симплекс с единичным ребром в евклидовом пространстве.
\end{itemize} 

\begin{enumerate}
\setcounter{enumi}{20}

\item Пусть у непрерывного отображения $f: B^n \to \mathbb{R}^n$ образ имеет размерность меньше, чем $n$. Докажите, что $f$ сдвигает какую-то точку на расстояние не меньше 1.

\item Пусть дано непрерывное отображение $f: B^n \to Z$ в метризуемое пространство размерности меньше, чем $n$. Докажите, что прообраз какой-то точки нельзя накрыть шаром радиуса меньше $1$.

\item Вычислите $\UW_{n-1}(B^n)$.

\item (Исследовательская задача) Чему равно $\UW_{n-2}(B^n)$?

\item (Исследовательская задача) Лемма Лебега о покрытиях влечёт, что $\UW_{n-1}(\square^n) = 1$. Правда ли, что $\UW_{d}(\square^n) = \sqrt{n-d}$?
    
\item Используя лемму Кнастера--Куратовского--Мазуркевича, покажите, что $\UW_{n-1}(\triangle^n) = 1/n$. 

\item (Исследовательская задача) Чему равны поперечники $\UW_{d}(\triangle^n)$?


\end{enumerate}

\end{document}


