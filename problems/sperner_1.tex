\documentclass[12pt]{article}

\pagestyle{empty}
\usepackage[utf8]{inputenc}
\usepackage[T2A]{fontenc}
\usepackage[russian]{babel}
\usepackage{cmap}
\usepackage{amsthm}
\usepackage{amsmath}
\usepackage{amssymb}
%\usepackage{units}
%\usepackage{fancyhdr}
%\usepackage{forloop}
\usepackage[pdftex]{graphicx}
\relpenalty=10000
\binoppenalty=10000

\renewcommand{\baselinestretch}{1.0}
\renewcommand\normalsize{\sloppypar}

\setlength{\topmargin}{-0.5in}
\setlength{\textheight}{9.1in}
\setlength{\oddsidemargin}{-0.4in}
\setlength{\evensidemargin}{-0.4in}
\setlength{\textwidth}{7in}
\setlength{\parindent}{0ex}
\setlength{\parskip}{1ex}

\newtheorem{theorem}{Теорема}
\newtheorem*{theorem*}{Теорема}
\newtheorem{lemma}{Лемма}
\newtheorem*{lemma*}{Лемма}
\newtheorem{problem}{Задача}

\newcommand{\HRule}{\rule{\linewidth}{0.5mm}}
\newcommand{\divrus}{\mathop{\raisebox{-2pt}{\vdots}}}

\renewcommand{\le}{\leqslant}
\renewcommand{\ge}{\geqslant}

\DeclareMathOperator{\gcdrus}{\text{НОД}}
\DeclareMathOperator{\lcmrus}{\text{НОК}}
\DeclareMathOperator{\maj}{maj}
\DeclareMathOperator{\per}{per}

\def\ra{\rightarrow}
\def\Ra{\Rightarrow}
\def\ov{\overline}
\def\CC{\mathbb{C}}
\def\RR{\mathbb{R}}
\def\NN{\mathbb{N}}
\def\QQ{\mathbb{Q}}
\def\ZZ{\mathbb{Z}}
\def\eps{\varepsilon}

\begin{document}

{
\HRule
 \normalfont \Large \center
   Листочек 1. Лемма Шпернера

\HRule
}



\begin{enumerate}
\setcounter{enumi}{0}
\item Квадрат, вершины которого покрашены в цвета 1, 2, 3, 4, триангулирован. Вершины триангуляции покрашены в те же 4 цвета, причём на стороне квадрата между вершинами цветов $i$ и $j$ использованы только цвета $i$ и $j$. Докажите, что в такой триангуляции можно найти не менее двух разноцветных треугольников триангуляции (треугольников, у которых вершины покрашены в три разных цвета).

  \item Вершины триангуляции треугольника покрасили какими-то цветами (их может быть много). Докажите, что найдётся хотя бы одно из двух --- разноцветный треугольник или одноцветный коннектор (связный подграф, задевающий все стороны исходного треугольника).

  \item (Дуальная лемма Шпернера) Пусть вершины триангуляции трёхмерного тетраэдра $\triangle^{3}$, грани которого назовём $F_0, F_1, F_2, F_3$, покрашены в цвета $0, 1, 2, 3$ так, что:
   \begin{itemize}
   \item относительная внутренность грани $F_i$ (т.е. грань без граничного треугольника) покрашена в цвет $i$; дополнительно гарантируется, что в этой относительной внутренности есть хотя бы одна вершина;

   \item относительная внутренность ребра $F_i \cap F_j$ (т.е. ребро без концов) покрашена в цвета $i,j$; дополнительно потребуем, чтобы в этой относительной внутренности была хотя бы одна вершина.

   \item вершина $F_i \cap F_j \cap F_k$ покрашена в любой из цветов $i,j,k$.
   \end{itemize}

   Докажите тогда, что существует элементарный тетраэдр разбиения с разноцветными вершинами.

   \emph{Замечание.} Технические ограничения можно опустить, если разрешить красить вершины в несколько цветов одновременно (в духе леммы Кнастера--Куратовского--Мазуркевича).

  \item (Частный случай теоремы Гейла--Бапата) Рассматриваются три различные раскраски вершин триангуляции треугольника $\triangle^{2}$. Первая раскраска использует цвета $1^{(1)}, 2^{(1)}, 3^{(1)}$, вторая --- цвета $1^{(2)}, 2^{(2)}, 3^{(2)}$, третья --- цвета $1^{(3)}, 2^{(3)}, 3^{(3)}$. Все три раскраски шпернеровские (в частности, вершина $i$ исходного треугольника покрашена в цвета $i^{(1)}, i^{(2)}, i^{(3)}$). Докажите, что существует элементарный треугольник разбиения, вершины которого покрашены в цвета $1^{(i)}, 2^{(j)}, 3^{(k)}$, где $(i, j, k)$ --- некоторая перестановка чисел $1,2,3$.

  \item Правильный треугольник разбит на 25 равных треугольников. Вершины триангуляции пронумеровали неким образом числами от 1 до 21. Докажите, что метки в вершинах какого-то ребра различаются хотя бы на 6.

  \item Два игрока играют в гекс на доске, имеющей форму ромба, разбитого на шестиугольники (см. рисунок во второй лекции). Докажите, что у первого игрока есть выигрышная стратегия.

  \item Пусть трёхмерный многогранник $P$ --- \emph{простой}, то есть в каждой вершине сходится три ребра. Допустим, $P$ покрыт объединением открытых множеств так, что никакая точка не накрыта четырьмя из них. Докажите, что какое-то из множеств покрытия задевает хотя бы четыре грани $P$.

  \item На единичном круге $B^2$ задано векторное поле $v: B^2 \ra \RR^2$, т.е. каждой точке $x = (x_1,x_2) \in B^2$ поставлен в соответствие вектор $v(x) = (v_1(x), v_2(x)) \in \RR^2$. Пусть известно, что поле нигде не обращается в ноль ($v(x) \neq (0,0)$), а также что поле $v(x)$ непрерывно зависит от точки $x$ (при малом изменении $x$ поле $v(x)$ меняется слабо). Рассматривается однократный обход граничной окружности $S^1 = \partial B^2$, запараметризованной при помощи функции $x(t), \ t\in[0,1]$, и измеряется количество оборотов вектора $v(x(t))$, когда $t$ пробегает от $0$ до $1$, а $x(t)$ пробегает граничную окружность. Докажите, что это число оборотов обязано быть нулевым.
  (\emph{Подсказка}: триангулируйте мелко круг и придумайте шпернеровскую раскраску, связанную с полем.)

  \item Пусть $g : B^3 \ra \RR^3$ --- непрерывная функция, такая что на границе шара $B^3$ имеет место неравенство $\langle g(x),x \rangle \geqslant 0$ (треугольными скобками обозначено скалярное произведение). Докажите, что $\exists x \in B^3 : g(x)=0$.

  \item (Очень частный случай теоремы Жордана) Пусть в прямоугольнике на плоскости нарисованы две кривые, заданные непрерывными функциями $\gamma_1, \gamma_2: [-1,1] \to [a,b]\times [c,d]$, причём $\gamma_1$ соединяет левую сторону прямоугольника с правой, а $\gamma_2$ соединяет верхнюю сторону с нижней. Докажите, что кривые пересекаются.

%  \item (Частный случай теоремы Пуанкаре--Миранды) На единичном квадрате $[0,1]\times[0,1]$ определены непрерывные вещественнозначные функции $f_1(x_1, x_2)$ и $ f_2(x_1, x_2)$. Известно, что $\forall x_2\in[0,1] \ f_1(0,x_2) < 0 < f_1(1,x_2)$ и $\forall x_1\in[0,1] \ f_2(x_1,0) < 0 < f_2(x_1,1)$. Докажите, что найдётся точка $(x_1, x_2) \in [0,1]\times[0,1]$, для которой $f_1(x_1, x_2) = f_2(x_1, x_2) = 0$.

  \item Докажите, что для любого непрерывного отображения $f : B^n \to \mathbb{R}^n$ ($B^n \subset \mathbb{R}^n$ --- единичный шар с центром в начале координат) найдётся точка $x \in B^n$, такая что $x = \lambda f(x)$, $0 < \lambda \le 1$.


  \item (Теорема Перрона--Фробениуса) Пусть матрица $(a_{ij})_{i,j=1}^n$ состоит из положительных чисел. Докажите, что у неё есть собственный вектор, состоящий из положительных чисел.
  (\emph{Подсказка}: это можно вывести из теоремы Брауэра.)

%  \item (Коннектор-лемма) Пусть вершины триангуляции $n$-мерного симплекса покрашены произвольно в цвета $1, 2, \ldots, n$. Докажите, что можно выбрать какой-то цвет $i$ так, что переходя только по рёбрам, соединяющим вершины цвета $i$, можно от любой грани добраться до любой.
%
%  \item (Гекс-лемма) Рассматривается мелко триангулированный $n$-мерный куб $I^n$. Каждому из $n$ игроков назначены две противоположные гиперграни куба (суммарно покрывающие всю границу куба). Далее происходит следующая игра: игроки по очереди выставляют фишки своего цвета в вершины триангуляции (в каждой вершине может находиться лишь одна фишка). Выигрывает тот, кто первым сможет соединить свои гиперграни цепочкой фишек своего цвета (в цепочке соседние вершины соединены ребрами триангуляции). Докажите, что игра не может закончиться ничьей.



%  \item (Теорема Кнастера--Куратовского--Мазуркевича) Пусть $n$-мерный симплекс $\triangle^n$ покрыт замкнутыми множествами $F_0, F_1, \ldots, F_n$: это означает, что $\triangle^{n} \subset \bigcup\limits_{i=0}^{n} F_i$. Пусть также известно, что $F_i$ не пересекает $i$-ую гипергрань симплекса (грани пронумерованы от $0$ до $n$). Докажите, что $\bigcap\limits_{i=0}^{n} F_i \neq \emptyset$.
%
%  \item (Теорема Лебега) Пусть $n$-мерный куб $I^n$ покрыт замкнутыми множествами, про которые известно, что ни одно из них не соединяет пару противоположных гиперграней (никакую пару противоположных граней нельзя соединить непрерывной кривой в рамках одного из множеств покрытия). Докажите, что существует точка куба, накрытая хотя бы $(n+1)$-кратно.

%  \item (Частный случай теоремы о бутерброде) Докажите, что в любом двумерном сэндвиче (сэндвич нарисован на плоскости) одним прямым сечением можно разрезать и хлеб, и котлету пополам. (Более общо: любые две конечные борелевские меры в $\RR^2$ можно поделить одной прямой пополам, при условии, что меры не концентрируются на прямых).
%
%  \item Выведите лемму Шпернера из теоремы Брауэра.

  \item (Частный случай теоремы Карасёва) Трёхмерный куб частично покрыт открытыми множествами, никакое из которых не задевает никакую пару противоположных граней куба. Пусть известно, что никакие три множества не имеют общей точки. Докажите, что какая-то компонента связности ненакрытой части куба пересекает четыре параллельных ребра куба.

  \item (Задача Н.~Н.~Константинова про возы) Из города $A$ в город $B$ ведут две не пересекающиеся дороги. Известно, что две машины, выезжающие по разным дорогам из $A$ в $B$ и связанные веревкой некоторой длины, меньшей $2\ell$, смогли проехать из $A$ в $B$, не порвав веревки. Могут ли разминуться, не коснувшись, два круглых воза радиуса $\ell$, центры которых движутся по этим дорогам навстречу друг другу?

  \item (Задача Н.~Н.~Константинова про альпинистов) Среди ровной степи стоит гора. На вершину ведут две тропы (считаем их графиками непрерывных функций, состоящими из конечного числа участков монотонности), не опускающиеся ниже уровня степи. Два альпиниста одновременно начали подъем (по разным тропам), соблюдая условие: в каждый момент времени быть на одинаковой высоте. Смогут ли альпинисты достичь вершины, двигаясь непрерывно?

  \item (Частный случай леммы Такера) Пусть триангуляция круга $B^2$ антиподальна на границе, т.е. такова, что вершины триангуляции, принадлежащие граничной окружности, при центральной симметрии с центром в нуле переходят в вершины триангуляции. Пусть также вершины триангуляции помечены метками $+1,+2,-1,-2$ так, что центрально симметричные вершины с границы помечены противоположными метками. Докажите, что в этой триангуляции найдётся \emph{диполь} --- ребро с противоположными метками на концах.

  \item Ковёр Серпинского --- подмножество плоскости, определяемое так. Стартуем с квадрата, который разрезаем на девять равных квадратов, и средних из них выбрасываем. Затем повторяем процесс для оставшихся восьми квадратов, и так далее счётное количество раз.
        \begin{itemize}
          \item Докажите, что топологическая размерность получившегося множества равна 1.
          \item Вычислите фрактальную размерность Минковского этого множества: $\lim\limits_{\eps \to 0} \frac{\ln N_\eps}{- \ln \eps}$, где $N_\eps$ --- минимальное число множеств диаметра $\eps$, которыми можно покрыть наше множество.
        \end{itemize}

%  \item (Частный случай теоремы фон Неймана о минимаксе) Пусть функция $f(x,y) : S\times T \ra \RR$ (где $S,T \subset \RR^n$ -- выпуклые замкнутые ограниченные множества) линейна по каждому аргументу. Можно, например, считать, что если $x = (x_1, \ldots, x_n)$, $y = (y_1, \ldots, y_n)$, то $f(x,y) = \sum\limits_{i,j} f_{ij} x_i y_j$ для некоторого набора чисел $f_{ij}, 1\le i,j\le n$. Докажите, что $\min\limits_y \max\limits_x f(x,y) = \max\limits_x \min\limits_y f(x,y)$.
%
%  \item Выигрыши в конечной игре двух игроков заданы матрицей (стратегии первого игрока -- в первом столбце, стратегии второго игрока -- в первой строке; в остальных клетках записаны выигрыши первого и второго игроков соответственно):
%
%\begin{center}
%\begin{tabular}{|c|c|c|}
%\hline
%&L&R\\
%\hline
%U&$3,1$&$0,2$\\
%\hline
%M&$1,2$&$1,1$\\
%\hline
%D&$0,4$&$3,1$\\
%\hline
%\end{tabular}
%\end{center}
%
%Найдите все равновесия Нэша (в том числе и в смешанных стратегиях) в этой игре.
%

%  \item
%  На плоскости дано $n$ красных и $n$ синих точек в общем положении. Докажите, что все точки можно разбить на разноцветные пары так, что отрезки, соответствующие парам, попарно не пересекаются.
%
%  \item
%  В трёхмерном пространстве дано $n$ красных, $n$ синих и $n$ зелёных точек в общем положении. Докажите, что все точки можно разбить на трёхцветные тройки так, что треугольники, соответствующие тройкам, попарно не пересекаются.
%
%  \item
%  На прямой дано $n$ красных и $n$ синих точек. Докажите, что все точки можно разбить на разноцветные пары так, что все отрезки, соответствующие парам, имеют общую точку.
%
%  \item
%  На прямой дано $2n$ красных и $2n$ синих точек. Докажите, что можно накрыть некоторым отрезком ровно половину красных и ровно половину синих точек.
%
%  \item
%  На плоскости дано $n$ красных, $n$ синих и $n$ зелёных точек. Докажите, что все точки можно разбить на трёхцветные тройки так, что все треугольники, соответствующие тройкам, имеют общую точку.
%
%  \item
%  Докажите, что если двумерная сфера $S^2$ покрыта тремя замкнутыми множествами, то хотя бы одно из них содержит пару антиподальных точек.
%
%  \item Приведите пример двух множеств на плоскости, которые нельзя одновременно рассечь одним прямолинейным разрезом так, чтобы площади кусков первого множества относились как 1:2 и площади кусков второго множества относились как 1:2.
%
%  \item
%  Докажите, что множество из $4n$ точек общего положения на плоскости двумя прямолинейными разрезами можно рассечь на 4 равные по мощности части.
%
%  \item
%  Докажите, что хлеб, сыр и колбасу, размазанные произвольным образом по плоскости, можно одновременно поделить пополам одной окружностью.

  \item Пусть плоскость триангулирована так, что любой треугольник триангуляции можно накрыть единичным кругом. Докажите, что тогда любой единичный круг задевает вершину триангуляции.


\end{enumerate}

\end{document}


